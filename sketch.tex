%%%%%%%%%%%%%%%%%%%%%%%%%%  ltexpprt_twocolumn.tex  %%%%%%%%%%%%%%%%%%%%%%%%%%%%%%%%
%
% This is ltexpprt_twocolumn.tex, an example file for use with the SIAM LaTeX2E
% Preprint Series macros. It is designed to provide two-column output.
% Please take the time to read the following comments, as they document
% how to use these macros. This file can be composed and printed out for
% use as sample output.

% Any comments or questions regarding these macros should be directed to:
%
%                 Donna Witzleben
%                 SIAM
%                 3600 University City Science Center
%                 Philadelphia, PA 19104-2688
%                 USA
%                 Telephone: (215) 382-9800
%                 Fax: (215) 386-7999
%                 e-mail: witzleben@siam.org


% This file is to be used as an example for style only. It should not be read
% for content.

%%%%%%%%%%%%%%% PLEASE NOTE THE FOLLOWING STYLE RESTRICTIONS %%%%%%%%%%%%%%%

%%  1. There are no new tags.  Existing LaTeX tags have been formatted to match
%%     the Preprint series style.
%%
%%  2. Do not change the margins or page size!  Do not change from the default
%%     text font!
%%
%%  3. You must use \cite in the text to mark your reference citations and
%%     \bibitem in the listing of references at the end of your chapter. See
%%     the examples in the following file. If you are using BibTeX, please
%%     supply the bst file with the manuscript file.
%%
%%  4. This macro is set up for two levels of headings (\section and
%%     \subsection). The macro will automatically number the headings for you.
%%
%%  5. No running heads are to be used for this volume.
%%
%%  6. Theorems, Lemmas, Definitions, Equations, etc. are to be double numbered,
%%     indicating the section and the occurrence of that element
%%     within that section. (For example, the first theorem in the second
%%     section would be numbered 2.1. The macro will
%%     automatically do the numbering for you.
%%
%%  7. Figures and Tables must be single-numbered.
%%     Use existing LaTeX tags for these elements.
%%     Numbering will be done automatically.
%%
%%  8. Page numbering is no longer included in this macro.
%%     Pagination will be set by the program committee.
%%
%%
%%%%%%%%%%%%%%%%%%%%%%%%%%%%%%%%%%%%%%%%%%%%%%%%%%%%%%%%%%%%%%%%%%%%%%%%%%%%%%%



\documentclass{article}

% Comment out the line below if using A4 paper size
\usepackage[letterpaper]{geometry}

\usepackage{ltexpprt}
\usepackage{hyperref}

\usepackage{amsfonts}           % \mathbb
\usepackage{amssymb}             % proof
\usepackage{xspace}             % \xspace
\usepackage{graphicx}
\usepackage{adjustbox}          % \adjustbox 
\usepackage{multicol}            
\usepackage{url}
\usepackage{enumerate, enumitem}  %enumerate environment with optional argument
\usepackage{subcaption}
\usepackage[usenames]{xcolor} % for coordinating edits
\newcommand{\sout}[1]{\st{#1}}
%% \usepackage{hyperref}
\usepackage{amsmath, hyperref, nicefrac}
\usepackage[capitalize]{cleveref}
\usepackage{thm-restate}
\usepackage{parskip}
\usepackage{mathtools}
\usepackage{algorithm}
\usepackage{algorithmic}


\colorlet{darkgreen}{green!45!black}

\newcommand{\R}{\mathbb{R}}
\newcommand{\set}[1]{\{#1\}}
\newcommand{\etal}{et al.\xspace}
\newcommand{\dist}{\text{dist}}
\newcommand{\eps}{\varepsilon}
\newcommand{\opt}{\text{OPT}}
\newcommand{\cost}{\text{cost}}
\newcommand{\calS}{\mathcal{S}}
\newcommand{\calP}{\mathcal{P}}
\newcommand{\calL}{\mathcal{L}}
\newcommand{\calR}{\mathcal{R}}
\newcommand{\calA}{\mathcal{A}}
\newcommand{\greedy}{\mathcal{A}}
\newcommand{\calT}{\mathcal{T}}
\newcommand{\ba}{\mathbf{a}}
\newcommand{\bb}{\mathbf{b}}
\newcommand{\bc}{\mathbf{c}}
\newcommand{\bd}{\mathbf{d}}
\newcommand{\bvf}{\mathbf{f}}
\newcommand{\bp}{\mathbf{p}}

\newcommand{\by}{\mathbf{p}}

\newcommand{\E}{\mathbb{E}}
\newcommand{\pr}{\mathbb{P}}
\newcommand{\calE}{\mathcal{E}}
\newcommand{\calI}{\mathcal{I}}
\newcommand{\calF}{\mathcal{F}}
\newcommand{\calX}{\mathcal{X}}
\newcommand{\calB}{\mathcal{B}}
\newcommand{\calC}{\mathcal{C}}
\newcommand{\bI}{\bar{I}_i}
\newcommand{\cand}{\mathbb{C}}
%\newcommand{\greedy}{\mathfrak{c}}
\newcommand{\A}{\mathcal{A}}
\newcommand{\poly}{\text{poly}}
\newcommand{\alg}{\greedy}
\newcommand{\centers}{\mathcal{C}}
\newcommand{\coreset}{\Omega}
\newcommand{\offset}{F}
\newcommand{\weight}{f}
\newcommand{\inner}{R_I}
\newcommand{\out}{R_O}
\newcommand{\main}{R_M}
\newcommand{\size}{\Gamma}
\newcommand{\polylog}{\text{polylog}}

\newcommand{\valuedelta}{\frac{\log^2 1/\eps}{2^{O(z\log z)}\min(\eps^2, \eps^z)}\left(k \log |\cand| + \log \log (1/\eps) + \log(1/\pi)\right)}



\begin{document}

%
\newcommand\relatedversion{}
\renewcommand\relatedversion{\thanks{The full version of the paper can be accessed at \protect\url{https://arxiv.org/abs/1902.09310}}} % Replace URL with link to full paper or comment out this line


%\setcounter{chapter}{2} % If you are doing your chapter as chapter one,
%\setcounter{section}{3} % comment these two lines out.

\title{\Large Variance Bound}
\author{}


\date{}

\maketitle

% Copyright Statement
% When submitting your final paper to a SIAM proceedings, it is requested that you include
% the appropriate copyright in the footer of the paper.  The copyright added should be
% consistent with the copyright selected on the copyright form submitted with the paper.
% Please note that "20XX" should be changed to the year of the meeting.

% Default Copyright Statement
%\fancyfoot[R]{\scriptsize{Copyright \textcopyright\ 20XX by SIAM\\
%Unauthorized reproduction of this article is prohibited}}

% Depending on which copyright you agree to when you sign the copyright form, the copyright
% can be changed to one of the following after commenting out the default copyright statement
% above.

%\fancyfoot[R]{\scriptsize{Copyright \textcopyright\ 20XX\\
%Copyright for this paper is retained by authors}}

%\fancyfoot[R]{\scriptsize{Copyright \textcopyright\ 20XX\\
%Copyright retained by principal author's organization}}

%\pagenumbering{arabic}
%\setcounter{page}{1}%Leave this line commented out.

\begin{abstract} \small\baselineskip=9pt \end{abstract}

Define the basic cost estimator for a solution $\calS$
$$ E_{\calS}:= \frac{1}{|\Omega|}\sum_{p\in \Omega} \frac{\cost(\greedy)}{\cost(p,\greedy)}\cost(p,\calS).$$

It's expectation is $\cost(\calS)$. We would like to show that the maximum error is less than $\varepsilon\cdot \cost(\calS)$ factor by considering the following expectation:

$$\mathbb{E}_{\coreset}\sup_{\calS}\left[\frac{|E_{\calS} - \cost(\calS)|}{\cost(\greedy)+\cost(\calS)}\right]$$

Using the symmetrization argument, we have 
$$\mathbb{E}_{\coreset}\sup_{\calS}\left[\frac{|E_{\calS} - \cost(\calS)|}{\cost(\greedy)+\cost(\calS)}\right]\leq O(1)\cdot \mathbb{E}_{\coreset}\mathbb{E}_{g}\sup_{\calS}\left[\frac{\frac{1}{|\Omega|}\sum_{p\in \Omega} \frac{\cost(\greedy)}{\cost(p,\greedy)}\cost(p,\calS) \cdot g_p}{\cost(\greedy)+\cost(\calS)}\right]$$


The way we analyse this so far is to first only condition on $\coreset$, with whatever properties it might have, and essentially only use the randomness of $g$ to bound the supremum:

$$ \mathbb{E}_{\coreset}\mathbb{E}_{g}\sup_{\calS}\left[\frac{\frac{1}{|\Omega|}\sum_{p\in \Omega} \frac{\cost(\greedy)}{\cost(p,\greedy)}\cost(p,\calS) \cdot g_p}{\cost(\greedy)+\cost(\calS)}\right] = \mathbb{E}_{\coreset}\mathbb{E}_{g}\sup_{\calS}\left[\left.\frac{\frac{1}{|\Omega|}\sum_{p\in \Omega} \frac{\cost(\greedy)}{\cost(p,\greedy)}\cost(p,\calS) \cdot g_p}{\cost(\greedy)+\cost(\calS)}~\right\vert~\coreset\right]$$

Whether or not we want to do chaining, we will eventually have to bound the variance of  $\frac{\frac{1}{|\Omega|}\sum_{p\in \Omega} \frac{\cost(\greedy)}{\cost(p,\greedy)}\cost(p,\calS) \cdot g_p}{\cost(\greedy)+\cost(\calS)}$. Since each is a Gaussian, the sum is a Gaussian with variance

$$ \sum_{p\in \Omega} \left(\frac{\cost(\greedy)}{\cost(p,\greedy)\cdot |\coreset|}\cdot\frac{\cost(p,\calS)}{\cost(\greedy)+\cost(\calS)}\right)^2$$

Here, I am no going to give a variance bound, assuming the following
\begin{enumerate}
\item All points in a cluster cost the same, up to constants
\item All clusters cost the same, up to constants
\item Conditioned on $\coreset$, we sample $\sum_{p\in \coreset\cap C} \frac{\cost(\greedy)}{\cost(p,\greedy)\cdot |\coreset|} = O(|C|)$ for all clusters $C$ of $\greedy$.
\item All clusters cost in $\calS$ roughly $2^i$ times their cost in $\greedy$.
\end{enumerate}

Define the clusters that satisfy condition 4 to be $L_i$ and let $|Li|=\alpha\cdot k$ be the number of clusters that satisfy the fourth condition. 

\begin{eqnarray*}
& & \sum_{p\in \Omega} \left(\frac{\cost(\greedy)}{\cost(p,\greedy)\cdot |\coreset|}\cdot\frac{\cost(p,\calS)}{\cost(\greedy)+\cost(\calS)}\right)^2 \\
\left(\frac{\cost(p,\calS)}{\cost(p,\greedy)}\approx 2^i\right)& \leq & \frac{1}{|\coreset|}\cost(\greedy)\cdot 2^i \cdot\left(\frac{1}{\cost(\greedy)+\cost(\calS)}\right)^2 \sum_{p\in \Omega} \frac{\cost(\greedy)}{\cost(p,\greedy)\cdot |\coreset|}\cdot\cost(p,\calS) \\
\text{Uniform Cost}& \leq & \frac{1}{|\coreset|}\cost(\greedy)\cdot 2^i \cdot\left(\frac{1}{\cost(\greedy)+\cost(\calS)}\right)^2 \sum_{C\in L_i}\frac{\cost(C,\calS)}{|C|}\sum_{p\in\Omega \cap C} \frac{\cost(\greedy)}{\cost(p,\greedy)\cdot |\coreset|} \\
\text{Condition on }\coreset& \leq & \frac{1}{|\coreset|}\cost(\greedy)\cdot 2^i \cdot\left(\frac{1}{\cost(\greedy)+\cost(\calS)}\right)^2 \sum_{C\in L_i}\frac{\cost(C,\calS)}{|C|}\cdot |C|\\
& \leq & \frac{1}{|\coreset|}\cost(\greedy)\cdot 2^i \cdot\left(\frac{1}{\cost(\greedy)+\cost(\calS)}\right)^2 \cost(\calS)\\
\end{eqnarray*}

This gives us a bound of $\frac{2^i}{|\coreset|}\leq \frac{\varepsilon^{-z}}{|\coreset|}$, if we ignore constants. Note that it doesn't depend on the space, however I doubt that using the space would give us much.

To get the alternative bound of $\frac{k}{|\coreset|}$, we can observe that $\alpha\geq 1/k$, as at least one cluster is in $L_i$, otherwise we wouldn't be considering it. Then using $\cost(\calS) \approx \alpha\cdot 2^i\cdot \greedy$

\begin{eqnarray*}
& &\cost(\greedy)\cdot 2^i \cdot\left(\frac{1}{\cost(\greedy)+\cost(\calS)}\right)^2 \cost(\calS) \\
&\leq & \cost^2(\greedy)\cdot 2^i \cdot\frac{1}{\cost^2(\greedy)(1+\alpha 2^i)^2} \cdot \alpha\cdot 2^i \\
&\leq &  \frac{\alpha \cdot2^{2i}}{\alpha^2 \cdot 2^{2i}} \leq  \frac{1}{\alpha} \leq k
\end{eqnarray*}

All of this is without the variance reduction $-q$ trick. But maybe this gives a bit of an idea what one could play with. Unless we modify the estimator (for example using the $q$ vector), this will cost us either an additional $k$ or and additional $\varepsilon^{-z}$ factor when doing chaining, assuming we can build the nets. In this case, building the nets is also doable (but also completely disjoint from obtaining the variance bound). 

\bibliographystyle{plain}
\bibliography{references}


%\begin{thebibliography}{99}
%
%%\bibitem{GUIDE}
%%R.~E. Bank, {\em PLTMG  users' guide, edition 5.0}, tech. report,
%%  Department of Mathematics, University of California, San Diego, CA, 1988.
%
%%\bibitem{HBMG}
%%R.~E. Bank, T.~F. Dupont, and H.~Yserentant, {\em The hierarchical basis
%%  multigrid method}, Numer. Math., 52 (1988), pp.~427--458.
%
%\bibitem{BANKSMITH}
%R.~E. Bank and R.~K. Smith, {\em General sparse elimination requires no
%  permanent integer storage}, SIAM J. Sci. Stat. Comput., 8 (1987),
%  pp.~574--584.
%
%\bibitem{EISENSTAT}
%S.~C. Eisenstat, M.~C. Gursky, M.~Schultz, and A.~Sherman, {\em
%  Algorithms and data structures for sparse symmetric gaussian elimination},
%  SIAM J. Sci. Stat. Comput., 2 (1982), pp.~225--237.
%
%\bibitem{GEORGELIU}
%A.~George and J.~Liu, {\em Computer Solution of Large Sparse Positive
%  Definite Systems}, Prentice Hall, Englewood Cliffs, NJ, 1981.
%
%\bibitem{LAW}
%K.~H. Law and S.~J. Fenves, {\em A node addition model for symbolic
%  factorization}, ACM TOMS, 12 (1986), pp.~37--50.
%
%\bibitem{LIU}
%J.~W.~H. Liu, {\em A compact row storage scheme for cholesky factors
%  using elimination trees}, ACM TOMS, 12 (1986), pp.~127--148.
%
%\bibitem{LIU2}
%\sameauthor , {\em The role of
%  elimination trees in sparse factorization}, Tech. Report CS-87-12,Department
%  of Computer Science, York University, Ontario, Canada, 1987.
%
%\bibitem{ROSE72}
%D.~J. Rose, {\em A graph theoretic study of the numeric solution of
%  sparse positive definite systems}, in Graph Theory and Computing, Academic  Press, New
%York, 1972.
%
%\bibitem{ROSE76}
%D.~J. Rose, R.~E. Tarjan, and G.~S. Lueker, {\em Algorithmic aspects of
%  vertex elimination on graphs}, SIAM J. Comput., 5 (1976), pp.~226--283.
%
%\bibitem{ROSEWHITTEN}
%D.~J. Rose and G.~F. Whitten, {\em A recursive analysis of disection
%  strategies}, in Sparse Matrix Computations, Academic Press, New York, 1976.
%
%\bibitem{SCHREIBER}
%R.~Schrieber, {\em A new implementation of sparse gaussian elimination},
%  ACM TOMS, 8 (1982), pp.~256--276.
%
%\end{thebibliography}
\end{document}

% End of ltexpprt.tex 